

\chapter{Metodikk}

\section{Rapportstruktur}
{\color{red}
Rapporten er bygget i en kronologisk rekkefølge. Den starter med et sammendrag av prosjektet og sa til en introduksjon av oppdragsgiver, oppgaven, gruppen og prosjektprosessen.\newline
Deretter følger det hvilken teori og metode prosjektgruppen har benyttet. Sa følger en del hvor resulta ter presenteres, diskuteres pa forskjellige nivå er og analyseres. Til slutt følger det en konklusjon.\newline
Avslutningsvis følger henvisninger til litteratur.\newline
Prosjektgruppen har besluttet a ikke strukturere rapporten basert pa teori, men tidligere bachelor- og masteroppgaver ved diverse universiteter og høgskoler.\newline
Denne rapporten inneholder ogsa elementer og struktur basert på en mal som emnean svarlig Gunnar Misund publiserte ved prosjektstart.
}

\section{Leveranser}
{\color{red}

HiØ stiller krav om ferdigstillelse av en hovedrapport og en rekke delinnleveringer: \newline
10. januar - Hjemmeside\newline
17. januar – Forprosjektrapport\newline
14. mars – Hoveddokument rev. 1\newline
25. april – Hoveddokumentet rev. 2\newline
22. mai – Hoveddokumentet rev. 3 endelig utgave\newline
2. juni – Opphenging av prosjektplakat\newline
5. juni – Muntlig framføring av prosjektet\newline
Våre leveranser til oppdragsgiver vil være oppnåelse av hovedmålet, som innebærer\newline
leveranse av en hovedrapport som inneholder informasjon om eksisterende teknologi\newline
relevant til vårt prosjekt, og spesifikasjoner om et eventuelt ferdig målesystem, samt å vise\newline
progresjon gjennom prosjektperioden. Sistnevnte kan oppnås ved å produsere en prototype,\newline
og eventuelt tilhørende programvare.\newline
}
\newpage

\section{Prosjekt Plan}
\vspace{10pt}
\textbf{Her er en grov plan vil laget for gjennomføring av projektet vårt.}\newline

\newcommand*{\TitleParbox}[1]{\parbox[c]{1.75cm}{\raggedright #1}}%
\begin{tabular}{|m{5cm}|m{5cm}|m{1cm}|}

\hline
\textbf{Aktivitet} & \textbf{Beskrivelse} & \textbf{Dato} \\
\hline
Forarbeid & Sette seg inn i oppgaven og forstå problemstillingen & 9/1 \\
\hline
Hjemmeside & Sette opp webside for prosjektet. & 9/1 \\
\hline
Forprosjektrapport & Levere forprosjektraport. & 16/1 \\
\hline
Møte med lærer i faget & Organisere et møte med en lærer som har et av fagene (IT1 eller IT2) & 19/1 \\
\hline
Informasjonsinnhenting/kildesøk & Innhente relevant informasjon om emnet (artikler, bøker, undersøkelse etc) & 23/1 \\
\hline
Intervjuguide & Produsere intervjuguide for kvalitative intervjuer.& 30/1 \\
\hline
Spørreundersøkelse & Spørreundersøkelse ferdigstilles & 30/1 \\
\hline
Godkjenning av spørreundersøkelse & Spørreundesøkelse godkjennes av oppdragsgiver & 06/2 \\
\hline
Utsending av spørreundersøkelse & Spørreundersøkelsen sendes ut til lærerne. & 06/2 \\
\hline
Alternativer til programvare & Finne alternativer til programvare (som ikke benyttes i dag.) & 13/2 \\
\hline
Teori og metode & Teori og metodekapitel ferdigstilles før første innlevering av hoveddok. & 13/3 \\
\hline
Første versjon av hoveddok & Første versjon av hoveddok leveres.& 13/3 \\
\hline
Konklusjon ferdig & Spørreundersøkelse er bearbeidet og, alternative programvarer funnet.& 24/4 \\
\hline
Andre versjon av hoveddok & Andre versjon av hoveddok leveres. & 24/4 \\
\hline
Korrekturlesning & Dokumentet er ferdig korrekturlest. &  21/5 \\
\hline
Vedlegg ferdig & Vedlegg er ferdig. & 21/5 \\
\hline
Innlevering av bachelor & Oppgaven leveres på eksamenskontoret & 21/5 \\
\hline
Prosjekplakat & Prosjektplakat er fedig produsert og printet. & 1/6 \\
\hline
Presentasjon av bachelorprosjekt & Prosjektet presenterer for oppdragsgiver, veileder og sensor. & 3,4,5 /6 \\
\hline
\end{tabular}

\newpage

