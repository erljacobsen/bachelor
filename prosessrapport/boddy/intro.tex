
\chapter*{Introduksjon}

Dataservice og Gyldendal Undervisning, heretter omtalt som oppdragsgiver,har et ønske om å kartlegge hvilke verktøy som i dag benyttes i undervisningen av fagene IT1 og IT2. Bachelorgruppen synes denne oppgaven virker interessant og utfordrende, samtidig som den bidrar til å øke kvaliteten i undervisningen i fagene IT1 og IT2 i den videregående skolen.\newline

% del 2  beskrivelse

\hspace{-17pt}Flere skoler har benyttet verktøy som Flash, PhotoShop, MySQL og Dreamweaver i IT1, mens Flash og ActionScript har blitt benyttet i IT2. Bruken av disse verktøyene er i endring, og oppdragsgiver ønsker en grundig kartlegging av hvilke verktøy de forsjellige skolene benytter i dag. I tillegg til kartlegging, ønsker oppdragsgiver en anbefaling fra prosjektgruppen angående hvilken programvare og hvilke verktøy som kan benyttes i lærebøkene. Formålet med dette prosjektet er at oppdragsgiver får en grundig ana-
lyse av hvordan utdanningen i praksis fungerer, med tanke på bruken av
de forskjellige verktøyene, samtidig som oppdragsgiver vil få en anbefaling angående hvilke verktøy som bør benyttes i undervisningen fremover. \newline

\hspace{-17pt}Selve oppgaven er avgrenset til å omhandle verktøy som benyttes, i form av programvare og programmeringsspråk. Det vil underveis i arbeidet bli vurdert om det er andre interessante aspekter som bør bli tatt med i oppgaven. Prosjektgruppen tar høyde for dette, da det kan dukke opp informasjon underveis som er nødvendig å ta med i det videre arbeidet med oppgaven.