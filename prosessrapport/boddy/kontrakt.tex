\begin{center}

\chapter{Kontrakter}
\vspace{20pt}
\begin{Huge}

\hspace{55pt}\textbf{Gruppekontrakt}\newline

\vspace{20pt}
Bacheloroppgaven

Avdeling for Informasjonsteknologi

Høgskolen i Østfold

Gruppe BO15G26

\end{Huge}
\vspace{40pt}
\begin{LARGE}
Markus Holmeset

Erlend Jacobsen

Jørgen Storm Nielsen

\vspace{40pt}
11. Desember 2014
\end{LARGE}

\end{center}
\newpage


\begin{large}
\begin{enumerate}
  \item \textbf{Demokrati}
  
Alle avgjørelser skal baseres på flertallet i gruppa. Medlemmene bør likevel prøve å komme fram til fullstendig enighet. 
  
  \item \textbf{Plikter}
  
Det er møte og forberedelsesplikt for alle. Ved sykdom eller annen gyldig fraværsgrunn skal ett eller flere av de andre medlemmene varsles så tidlig som mulig. Medlemmene forplikter seg også til å levere avtalt arbeid til rett tid, eller si fra i god tid hvis forsinkelser oppstår.

  \item \textbf{Møtetider}
  
Møtetider bør avtales slik at ingen i gruppen hindres i foreberedelser eller oppmøte.
  
  \item \textbf{Utgifter}
  
Alle utgifter i prosjektet (reising, kopiering, inngangspenger, tidsskrifter etc.) skal deles likt mellom medlemmene. Utgiftene skal kunne dokumenteres. 
  
  \item \textbf{Tvister}
  
Uenigheter og problemer bør bli behandlet og løst i plenum, internt i gruppen. Hvis dette slår feil, skal veileder kontaktes.

  
  \item \textbf{Arbeidsdeling}
  
Alt arbeid skal deles så likt og rettferdig som mulig. Alle har rett til å påpeke forhold de mener er urettferdige. Totalt må hvert medlem bruke ca. 500 timer på prosjektet. Dette inkludert alt relatert arbeid, slik som f.eks. møter med oppdragsgiver og veileder. Det skal føres individuelle timelister som dokumenterer tidsbruken.
  
  \item \textbf{Samarbeid}
  
I en gruppe gjelder en for alle, alle for en. Det er viktig at alle forsøker å dele tid, arbeid, erfaringer og kunnskap. Det er en selvfølge å hjelpe til hvis noen står fast.

  
  \item \textbf{Skjevfordeling}
  
Ved en åpenbar skjevfordeling kan det være aktuelt å gi ulike karakterer i gruppen. Dette vil i så fall være etter gruppens eget ønske. Gruppen må fange opp slike problemer så raskt som mulig, og prøve å løse de umiddelbart, gjerne i samråd med veileder og evt. fagansvarlig.

  \item \textbf{Kontraktsbrudd}
  
Alvorlige og/eller gjentatte regelbrudd kan føre til eksklusjon fra gruppen. Dette må i så fall skje i samråd med veileder og fagansvarlig. 

\end{enumerate}
  
\end{large} 

\vspace{40pt}

\begin{center} \textbf{Underskrifter:}

\vspace{20pt}
Erlend Jacobsen \hspace{50pt} Jørgen Storm Nielsen \hspace{50pt} Markus Holmeset
\end{center}

\newpage





\begin{center}

\begin{Huge}

\hspace{55pt}\textbf{Prosjektkontrakt}\newline

\vspace{20pt}
Bacheloroppgave

«Programvare i fagene Informasjonsteknologi 1 og 2»

Avdeling for Informasjonsteknologi

Høgskolen i Østfold
\vspace{20pt}
\end{Huge}

\begin{LARGE}
\textbf{Gruppe BO15G26}
Jørgen Storm Nielsen

Markus Holmeset

Erlend Jacobsen

\vspace{20pt}
\textbf{Oppdragsgiver:}

Gyldendal Forlag/Dataservice

Øystein Falch

\vspace{20pt}
\textbf{Veileder:}

Tom Heine Nätt



\vspace{40pt}
11. Desember 2014
\end{LARGE}

\end{center}

\newpage

\large
\vspace{40pt}
\hspace{-17pt}\textbf{Prosjektbeskrivelse}\newline
Dataservice as og Gyldendal Undervisning utvikler i
samarbeid læremidler til fagene informasjonsteknologi 1 og 2
i den videregående skolen (IT 1 og IT 2)\newline

\hspace{-17pt}I IT 1 har mange skoler benyttet Flash, PhotoShop, MySQL
og Dreamweaver som verktøy. I IT 2 har Flash vært brukt
som verktøy for å lage multimedieapplikasjoner, mens
ActionScript har vært benyttet som programmeringsspråk.\newline
Bruken av verktøy er under endring og prosjektet skal ha to
hovedoppgaver:
\begin{enumerate}
\item Kartlegge hvilke verktøy og programmeringsspråk
som i dag benyttes til fagene

\item Komme fram til anbefalte forslag til programvare som
kan benyttes framover

\end{enumerate}

\hspace{-17pt}\textbf{Studentgruppen}\newline
Studentgruppen har selv det fulle ansvar når det gjelder organisering og gjennomføring av prosjektet. Ikke minst innebærer dette å følge opp tidsfrister. Kontakt med oppdragsgiver og veileder er også studentenes ansvar. De må også holde seg oppdatert på beskjeder og meldinger fra fagansvarlig (som regel pr. epost). Studentene skal selv fordele arbeidet i gruppa, og eventuelt velge prosjektleder (kan byttes på underveis hvis ønskelig).
Gruppen har rett til veiledning fra en eller flere av HiØ/IT sine fagansatte tilsvarende ca. en halv time pr. uke gjennom hele semesteret. Studentene skal også ha fri tilgang til program- og maskinvare og annet nødvendig utstyr. Er det behov for reiser og liknende, skal dette dekkes etter avtale med oppdragsgiver og/eller HiØ/IT.\newline

\hspace{-17pt}\textbf{Oppdragsgiver}\newline
Det er viktig at oppdragsgiver er innforstått med rammene rundt bacheloroppgaven. Oppdragsgiver må ikke definere en oppgave som ellers ville bli utført av eksterne konsulenter o.l. Det vil alltid være en risiko for at studentene ikke kommer helt i mål, eller leverer noe som ikke nødvendigvis er ferdige, operative løsninger.
Oppdragsgiver skal gi rammene for prosjektet, dvs. hjelpe til med å definere formål, leveranser, og evt. metode. Det er svært viktig å gi studentene nok informasjon, og evt. tilgang til utstyr, programvare o.l.
Oppdragsgiver må være forberedt på å svare på spørsmål på kort varsel, og sørge for å være oppdatert på prosjektets framdrift. Det kan også være aktuelt å hjelpe til med å revidere og endre prosjektplanen underveis.
Det er vanlig å ha regelmessige møter med gruppa, med en frekvens som passer prosjektets egenart.\newline

\hspace{-17pt}\textbf{Veileder}\newline
Veileder skal hjelpe til med: 1) Gjennomføring og dokumentasjon, og 2) Faglige spørsmål (i de tilfellene der det faglige blir for smalt og spesielt, må studentene enten skaffe faglig hjelp fra oppdragsgiver, eller fra annet hold).\newline
\newpage

\hspace{-17pt}\textbf{Rettigheter}\newline
Deltagere med rettigheter i prosjektarbeidet er ideforfatter, utøvende medarbeidere og oppdragsgiver.
Ideforfatter kan være veileder, en eller flere studenter, oppdragsgiver eller en kombinasjon av disse.
Utøvende medarbeidere er studentene og i noen grad veileder.
Oppdragsgiver kan enten være en ekstern institusjon/bedrift eller en avdeling ved Høgskolen i Østfold.\newline
Følgende rettigheter er knyttet til prosjektdeltagelse:
\begin{itemize}
\item Opphavsrett: Rett til benevnelse ved publikasjon og eiendomsrett til originalt åndsverk (som sikres ved publikasjon). Opphavsrett innehas av ideforfatter og utøvende medarbeidere.

\item Disposisjonsrett: Rett til videreutvikling av et avsluttet arbeid. Denne innehas av oppdragsgiver.

\item Eiendomsrett: Rett til oppbevaring og omsetning av et avsluttet arbeid og innehas av oppdragsgiver.

\end{itemize}
Avvik fra disse prinsippene kan avtales før godkjenning av prosjekter etter ønsker fra oppdragsgiver. Det skal da lages en avtale som inngås av alle deltagerne i prosjektet. Tvister om rettigheter avgjøres etter deltagernes samtykke av avdelingsstyret, subsidiært ved forfølgelse i det alminnelige rettsvesen.\newline
Hvis oppdragsgiver ikke ønsker at resultatene eller deler av disse skal offentliggjøres, må dette avklares ved oppstart av prosjektet.

\vspace{40pt}

\begin{center}\textbf{Underskrifter:}

\vspace{20pt}
Erlend Jacobsen \hspace{50pt} Jørgen Storm Nielsen \hspace{50pt} Markus Holmeset

\vspace{80pt}
Øystein Falch \hspace{230pt} Tom Heine Nätt

\end{center}


% // insert GFX




