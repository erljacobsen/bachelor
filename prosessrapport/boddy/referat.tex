\chapter{Møtereferat}

I gjennom prosjekt prosessen har vi hatt endel møter med blant annet oppdragsgiver og veileder. Dette kappittele vil ta for seg referat ifra møtene. Referatene ble brukt i prosjektet for å få en oversikt av hva som ble gjennomgått på møtene. Det var veldig nyttig å kunne hente frem alle stikkord og ideer som ble ytret på møtene. 
Det er alltid no som man ikke husker etter et møte siden det blir gjennomgått mye og hukommelsen strekker ikke alltid til.\newline
\vspace{30pt}


\hspace{-17pt}

\hspace{-17pt}\textbf{Så her kommer vedlagte møtereferat sortert etter dato:}

\newpage


% // Start møte med Oppdragsgiver


\section{Møte 12/11/14}

\vspace{30pt}
\textbf{Tid:}\newline
10:00\newline 

\hspace{-17pt}\textbf{Sted:}\newline 
Dataservice , Halden
\newline

\hspace{-17pt}\textbf{Deltagere:}\newline 
Tom Heine Nätt, Erlend Jacobsen, Markus Holmeset,\newline
Jørgen Storm Nielsen og Øystein Falch
\newline

\hspace{-17pt}\textbf{Fra møtet:}\newline
Det er viktig å tenke tidlig og planlegge arbeidet godt. Vi får Brukernavn og passord for lærersidene til fagene IT1 og IT2. Vi må lage en liste av kriterier, ut i fra læreplanen og tidligere eksamener. Informasjon kan finnes på Utdanningsdirektoratet sine hjemmesider.\newline
Hva kan HTML og Javascript gjøre ?\newline
Hva gjør lærerene i dag og hva tenker de fremover. Det vil være bra å sende ut en landsomfattende spørreundersøkelse for begge fagene. Øystein har liste med mailadresse til alle lærerene som underviser i fagene. Kan være lurt å ha et kort intervju med en eller flere lærere i nærmiljøet for å få et innblikk i hvordan fagene foregår.\newline
Det er veldig viktig å stille riktige og upartiske spørsmål. Først og fremst må man vite hva man vil ha svar på. IT 1 og IT 2 trenger hver sin undersøkelse.\newline

\vspace{20pt}

\hspace{-17pt}\textbf{Videre arbeid:}\newline
Starte planllegingsfase av prosjektet.\newline
Utforme hypotese.\newline
Starte å tenke tidlig, det er viktig for en god oppgave.\newline







\newpage

%  //   tirsdag 6 januar Møtereferat

\section{Møte 06/01/15}

\vspace{30pt}
\textbf{Tid:}\newline
10:00\newline 

\hspace{-17pt}\textbf{Sted:}\newline 
Høgskolen i Østfold.\newline

\hspace{-17pt}\textbf{Deltagere:}\newline 
Tom Heine Nätt, Erlend Jacobsen, Markus Holmeset og Jørgen Storm Nielsen\newline

\hspace{-17pt}\textbf{Agenda:}

\begin{itemize}
\item Hjemmeside til prosjektet. (Ferdigstilles)
\item Ny versjon av gruppekontrakt?
\item Starte med forprosjektrapport.
\item LaTex
\item Insamling av kilder (starte på dette)

\end{itemize}


\hspace{-17pt}\textbf{Fra møtet:}\newline
Hjemmesiden er “live” og godkjent av Tom Heine.\newline
Dokument for forprosjektrapport er opprettet i google drive. Denne skrives inn i LaTex når den er mer eller mindre ferdig. Jørgen og Erlend laster ned LaTex og tar oppgaven med å lære seg dette. Markus og Erlend har lett litt i Google Scholar etter litteratur på området, funnet et par interessante kilder (samlet i eget dokument på drive)\newline


\hspace{-17pt}\textbf{Videre arbeid:}\newline
Jørgen skriver videre på forprosjektrapport.\newline
Denne ferdigstilles etter møtet med Tom Heine på mandag 12/1.\newline 
Jørgen og Erlend setter seg inn i LaTex.\newline
Markus fortsetter å finne litteratur.\newline


\hspace{-17pt}\textbf{Konkrete oppgaver:}\newline
Forprosjektrapport: Ferdig 12/1\newline

\newpage

%  //   freadg 9 januar Møtereferat

\section{Møte 09/01/15}

\vspace{30pt}
\textbf{Tid:}\newline
09:15\newline

\hspace{-17pt}\textbf{Sted:}\newline 
Høgskolen i Østfold.\newline

\hspace{-17pt}\textbf{Deltagere:}\newline 
Tom Heine Nätt, Erlend Jacobsen, Markus Holmeset og Jørgen Storm Nielsen\newline

\hspace{-17pt}\textbf{Fra møtet:}\newline
Nettsiden er ferdig. Godkjent av TH.\newline
Fikk utdelt bøker fra programfaget IT1 som vi bør “skumme” igjennom for litt insikt i hva faget egentlig dreier seg om.\newline

\vspace{20pt}

\hspace{-17pt}\textbf{Videre arbeid:}\newline
Vi å ta kontakt med kommuner ang Adobepakken (pris, kontrakt osv). \newline
Også kontakte eventuelle lærere om mulig møte, intervju og undervisningstime.\newline
Forprosjektrapporten må benynnes på. Skal være ferdig til Fredag. \newline
Komme igang med innsamling av litteratur. \newline
Strukturere videre arbeid. Fordele oppgaver.\newline

\vspace{20pt}

\hspace{-17pt}\textbf{Oppgaver:}\newline
Forprosjektrapport \newline
Samle informasjon\newline

\newpage

%  //   mandag 12 januar Møtereferat

\section{Møte 12/01/15}

\vspace{30pt}
\textbf{Tid:}\newline
09:15\newline

\hspace{-17pt}\textbf{Sted:}\newline 
Høgskolen i Østfold.\newline

\hspace{-17pt}\textbf{Deltagere:}\newline 
Tom Heine Nätt, Erlend Jacobsen, Markus Holmeset og Jørgen Storm Nielsen\newline

\hspace{-17pt}\textbf{Fra møtet:}\newline
\underline{1.Del med Tom Heine Nätt}\newline

\hspace{-17pt}Gjennomgang av hva som har blitt gjort den siste uken. Forprosjektrapport er startet. Bør ikke brukes så ALT for mye tid på denne. Det viktigste er formuleringen av mål og hva som skal gjøres ifølge TH.\newline 
Stort fokus på veien videre. Opprette kontakt med kommuner. \newline
Tipset om Trello for å organisere arbeidet. \newline

\underline{2. Del med gruppen} \newline

\hspace{-17pt}Arbeid med forprosjektrapport. Skal ferdigstilles i løpet av uka (helst i dag, senest fredag kveld)\newline
Arbeid med videre plan. Denne MÅ ferdigstilles asap. \newline

\vspace{20pt}

\hspace{-17pt}\textbf{Videre arbeid:}\newline
Ferdigstille forprosjektrapporten. Legge denne ut på nettsiden og sende kopi til TH.\newline
Ta kontakt med Edgar ang læreplan. \newline
Sende mail til lærere ang møte. \newline
Ta kontakt med kommuner. \newline

\vspace{20pt}

\hspace{-17pt}\textbf{Oppgaver:}\newline
Forprosjektrapport\newline
\newpage

%  //   mandag 19 januar Møtereferat


\section{Møte 19/01/15}

\vspace{30pt}
\textbf{Tid:}\newline
09:15\newline

\hspace{-17pt}\textbf{Sted:}\newline 
Høgskolen i Østfold.\newline

\hspace{-17pt}\textbf{Deltagere:}\newline 
Tom Heine Nätt, Erlend Jacobsen, Markus Holmeset og Jørgen Storm Nielsen\newline

\hspace{-17pt}\textbf{Fra møtet:}\newline
Gjennomgang av for forprosjektrapport og utbedringer som måtte gjøres.\newline
Gjennomgang av plan videre for arbeidet og kontakt av lærer ved Fredrik II VGS.\newline
Åpne for ting som dukker opp, der vi har sagt at vi ikke skal ta med dette… \newline
litt utdypning om informasjonsinnhenting.\newline
Intervju må forberedes og vi må teste spørreundersøkelsen.\newline
Husk at eventuelt nye ting skal læres ved bytte av programvare, Vanskelig for lærerne og.\newline
Pedagogikk, enkelt å komme i gang VS yrkesrelevant. (bedrifter, hvor mye har dette å si at eleven kan ditt og datt?)\newline


\vspace{20pt}

\hspace{-17pt}\textbf{Videre arbeid:}\newline
Informasjons innhenting , kilder og referanser.\newline
Starte å se på hovedraport.\newline
Intervju spørsmål med Fredrik II VGS.\newline

\hspace{-17pt}\textbf{Oppgaver:}\newline
Sende forprosjektrapporten etter den har blitt rettet.\newline


\newpage



%  //   mandag 9 februar Møtereferat

\section{Møte 09/02/15}

\vspace{30pt}
\textbf{Tid:}\newline
09:15\newline

\hspace{-17pt}\textbf{Sted:}\newline 
Høgskolen i Østfold.\newline

\hspace{-17pt}\textbf{Deltagere:}\newline 
Tom Heine Nätt, Erlend Jacobsen, Markus Holmeset og Jørgen Storm Nielsen\newline

\hspace{-17pt}\textbf{Fra møtet:}\newline
Gjennomgang av struktur på hovedrapporten. Tilbakemelding var at vi burde prøve å skille rapporten i to deler, en med all fakta og det viktige vi har funnet, og en rapport med arbeidsprosessen samt viktige vedlegg. 
Grei start på oppsettet og når vi får flyttet noen av kapittelene vil det bli bedre.\newline

\hspace{-17pt}Struktur i gruppen og arbeids effektivitets tips. I stede for at alle på gruppen jobber feller på ting kan man dele arbeids oppgaver inn i tre deler. Bruke Trello til en digital oppslagstavle for å få oversikt over hvilke oppgaver som skal utføres. Også kan man ha forskjellige roller som Iverksetter , Kordinator og Kontrollør.\newline

\vspace{20pt}

\hspace{-17pt}\textbf{Videre arbeid:}\newline
Dele opp rapporten i to deler.\newline
Rette litt på rapport strukturen.\newline
Effektivisere arbeidet innad i gruppen.\newline
Starte å lage spørreundersøkelsen som skal sendes rundt.\newline



\newpage


%  //   mandag 23 februar Møtereferat

\section{Møte 23/02/15}

\vspace{30pt}
\textbf{Tid:}\newline
09:15\newline

\hspace{-17pt}\textbf{Sted:}\newline 
Høgskolen i Østfold.\newline

\hspace{-17pt}\textbf{Deltagere:}\newline 
Tom Heine Nätt, Erlend Jacobsen, Markus Holmeset og Jørgen Storm Nielsen\newline

\hspace{-17pt}\textbf{Fra møtet:}\newline
Gjennomgang av strukturen på prosessrapporten, konkludert med at det er bedre å skille rapporten i to deler. En med all fakta og det viktige vi har funnet, og en rapport med arbeidsprosessen samt viktige vedleg. Må ha en konklusjon i prosessrapporten.

Tilbakemelding på vår første utkast av spørreundersøkelsen. Spørsmålene må utdypes og formuleres bedre. Spørsmålene må lages ut i fra hva vi har lyst til å finne ut. Må også få nøytral tilbakemelding om lærerene egentlig ønsker å bytte.\newline

\hspace{-17pt}\textbf{mulige spørsmål:} \newline

\begin{itemize}

\item Hvor mange bruker eget materiale i fagene.
\item Hvilken programmvare benyttes i opplæringen.
\item Nytteverdien av programmvaren.
\item Hvilke bøker blir benyttet, en side pr.bok med innspill / kommentar felt.
\item Hva gjør lærerene fornøyde , er de fornøyde med dagens situasjon.
\item Hvor flinke er lærerene i det de underviser.

\end{itemize}

\hspace{-17pt}Få med litt teori om spørmål og spørreundersøkelse.\newline 
Det er viktig med riktig inndeling av undersøkelse.\newline

\vspace{20pt}

\hspace{-17pt}\textbf{Videre arbeid:}\newline
Referat i fra møtet med øystein.\newline
Kontinuelig arbeid på prosessrapporten.\newline
Intervju med Stabekk VGS.\newline
formulere og utbedre spørreundersøkelsen.\newline
\newpage



% // møte 2 Mars



\section{Møte 02/03/15}

\vspace{30pt}
\textbf{Tid:}\newline
09:15\newline

\hspace{-17pt}\textbf{Sted:}\newline 
Høgskolen i Østfold.\newline

\hspace{-17pt}\textbf{Deltagere:}\newline 
Tom Heine Nätt, Erlend Jacobsen, Markus Holmeset og Jørgen Storm Nielsen\newline

\hspace{-17pt}\textbf{Fra møtet:}\newline
\hspace{-17pt}\underline {Tilbakemelding på spørsmål som skal sendes ut:}

\begin{itemize}
\item \textbf{Læremiddler:} Ikke sett kryss om du ikke benytter…. :
\item \textbf{Programvare:} Er det noe som ikke samsvarer med bøkene
\item \textbf{Programvare:} Hvilke funksjoner benytte i programmvaren eller ikke benyttes.
\item \textbf{Programvare:} Hvordan fyller programmet læreplanens mål.
\item \textbf{Programvare:} Hva vektlegger de ? “pris” “relevans” “brukervennelighet” “tilgjengelighet”
\end{itemize}

\hspace{-17pt}\underline {Gennerelle komentarer til undersøkelsen:}

\begin{itemize}
\item Section Header for å dele opp spørsmålene
\item Vite om andre ting folk bruker.
\item Ikke for mye tekst så det blir vanskelig å svare eller kjedelig
\item Økonomisk situasjon i forhold til bytte av læremiddler.
\item Hva taler for å bytte, er det motivasjon for kunskapsoppdatering / opplæring
\end{itemize}

\hspace{-17pt}Må også tenke på hvorfor de svarer som de gjorde. Vet alle lærerene om alt som finnes ? Mange gjør nok bare det de blir fortalt uten å være kritiske.\newline
Viktig også å sjekke skrivefeil i undersøkelsen ! Den må virke proff.\newline

\vspace{20pt}

\hspace{-17pt}\textbf{Videre arbeid:}\newline
Sende spørsmål til Øystein Falch 02.03.2015 med liten tekst om hva vi har gjort.\newline
Ledetekst til undersøkelsen må lages og fremheve frist på 2 uker og ikke for mye vekt på bytte av læremiddler.\newline
Hvert medlem må finne 5 feil med undersøkelsen. (for å utbedre den) \newline
Ringe lærere for intervju denne uken.\newline

\newpage



\section{Møte 05/03/15}


\vspace{30pt}
\textbf{Tid:}\newline
13:00\newline 

\hspace{-17pt}\textbf{Sted:}\newline 
Dataservice , Halden
\newline

\hspace{-17pt}\textbf{Deltagere:}\newline 
Erlend Jacobsen, Markus Holmeset,\newline
Jørgen Storm Nielsen og Øystein Falch
\newline

\hspace{-17pt}\textbf{Fra møtet:}\newline
Undersøke budsjettering på programmvare og læremiddler på enkelte utvalgte skoler.
Er det felles eller delte budsjetter?\newline
Spørreundersøkelsen er en stor del av oppgaven så det ble lagt mest vekt på denne under møtet.\newline


\textbf{Gjennomgang av spørreundersøkelsen med oppdragsgiver:}
\begin{enumerate}
\item Legge inn “Navn på Skole” under hvilke fylker de er i fra, for å få bedre og mer spesifikk data. Dette gjør at vi må fjærne anonymitet.
\item Undervisning med stor U. 
\item Alternativet “Snarvei til Access 2003” kan fjærnes.
\item Lage en droppliste for hvilken programvare som benyttes og for punkt 6. under.
\item Endre “ Hvilke funksjoner i programvaren benytter du til å dekke målene i læreplanen?” til “Hvor har du valgt denne programmvaren?” både i IT-1 og IT-2.
\item Legge til “Hvilken programvare tenke du å benytte neste år?”.
\item Legge til “CC” i listen under punkt 6.
\item Legge til alternativ “Annet” med eget skrivefelt.
\item Fjærne “ Valgkriterie i bunnen av skjemaet”.
\item Undersøkelsen bør ha en frist på 1 uke.
\end{enumerate}

Utover dette synes oppdragsgiver at det var gjort en bra jobb til nå, og oppfordrer til like god innsats videre.

\vspace{20pt}

\hspace{-17pt}\textbf{Videre arbeid:}\newline
Lage en tekst som skal følge undersøkelsen.\newline
Oppgave sendes ut 09/03/2015.\newline
Data motatt behandles etter 16/03/2015.\newline

\newpage




\section{Møte 11/03/15}


\vspace{30pt}
\textbf{Tid:}\newline
09:00\newline 

\hspace{-17pt}\textbf{Sted:}\newline 
Høgskolen i Østfold
\newline

\hspace{-17pt}\textbf{Deltagere:}\newline 
Tom Heine Nätt, Erlend Jacobsen, Markus Holmeset og Jørgen Storm Nielsen\newline
\newline

\hspace{-17pt}\textbf{Fra møtet:}\newline
Må forberede svar på spørreundersøkelsen. Hvordan skal dataen presanteres.
Vi burde slå sammen skoler som en helhet.\newline
Hvor mange skoler har vi fått svar i fra i forhold til hvor mange som finnes.\newline
Vi må få en oversikt over hvilken programmeringsspråk som brukes på de forskjellige skolene.\newline

\hspace{-17pt}Kapitler må skrives i riktig rekkefølge for å få en god troverdighet.\newline
Få en oversikt over alt som skal gjøres og skrives.\newline
Hvordan skal rapporten dvs. produktet vårt se ut når det er ferdig.\newline
Vi må finne ut hvordan vi skal sy sammen alt det som er samlet inn til en bra og uvinklet rapport.\newline
Hva skal samtaler med Videregående skole bringe inn, hva skal denne informasjonen brukes til.\newline
Vi må gjøre en del mer forarbeid til rapporten’s helhet.\newline
Det burde lages en detaljert struktur på oppgaven med ledetekst.\newline
Husk at undersøkelsen burde kun være 30"\%"av data innhentet.\newline


\vspace{20pt}

\hspace{-17pt}\textbf{Videre arbeid:}\newline
Bearbeide data.\newline
Få ferdig første utkast av rapporten.\newline
Sende første utkast til Tom Heine på mandag.\newline

\newpage


\section{Møte 16/03/15}

\textbf{Tid:}\newline
14:45\newline 

\hspace{-17pt}\textbf{Sted:}\newline 
Høgskolen i Østfold
\newline

\hspace{-17pt}\textbf{Deltagere:}\newline 
Tom Heine Nätt, Erlend Jacobsen, Markus Holmeset og Jørgen Storm Nielsen\newline

\hspace{-17pt}\textbf{Fra møtet:}\newline
Forsiden må endres til noe renere, avdeling remmen byttes til Informasjonsteknologi.\newline
Teksten burde rettskrives med en gang ellers glemmes det.\newline
Hvis vi skal dele rapporten må vi gjøre dette gjennomført!\newline
Problemstilling burde bli nøye og ordentelig skrevet.\newline
Starten av rapporten burde inneholde problemet vi skal løse.\newline
Husk å markere tekst som ikke er endelig i egen farge.\newline
Eksamenoppgaver og kravsprdifiksdjon er ikke under Teori kapittelet siden disse er essensielle for vår oppgave.\newline
Pedagogikk, hvordan foregår dette i videregående skoler.\newline
Del opp hvert intervju i sub-section, og ha intervju som en section.\newline
Dagens programvare og alternativer til programvare er fakta.
Hvor mange skoler og elever tar fagene. Skal hjelpe oss å se trender.\newline

\hspace{-17pt}\textbf{Teori kapittelet:} skal inneholde kun ting folk med innsikt kan hoppe over uten å miste noe av innholdet i rapporten.\newline
\hspace{-17pt}\textbf{Metode kapittelet:} skal beskrive hvordan vi overordnet angrep problemet. Vi skal ikke skrive en teoribok om metoder, men hvordan vi brukte eksisterende metoder for å løse vårt problem.\newline
\hspace{-17pt}\textbf{Gjennomføring kapittelet:} skal ligge imellom metode og resultat, så det skal deles inn i de kapittelene i stede for å være et eget kapittel.\newline
\hspace{-17pt}\textbf{Resultat og Analyse kapittelet:} bør deles i to deler. Analyse skal vi på en måte diskutere og argumentere . og i tilslutt trekke en konklusjon ut i fra dette.\newline

\hspace{-17pt}Spesifiser hva kapittelene skal omhandle og fylle ut litt tekst om dette.\newline
Vi må være forsiktig med å ikke mene for mye i rapporten, i hvertfall ikke før konklusjonen.\newline
For øyeblikket virker hele rapporten basert på en undersøkelse.\newline
Mangler endel i kapittel 5. Kun fakta må med i rapporten.\newline
Hvilke avtaler har kommunene med programvar produsentene.\newline
Vi burde også ta en titt på hva som gjøres i andre land.\newline
Resultatet i fra undersøkelsen ser noe merkelig ut og burde sjekkes.\newline
Teori om utdanning i Videregåendeskole.
Teori om utdanning gennerlet.
IT fagene's historie.\newline

\hspace{-17pt}\textbf{Videre arbeid:}\newline
Jobbe med struktur av dokumentet.\newline
Forberede analyse av data som er innsamlet.\newline

\newpage

\section{Møte 25/03/15}

\textbf{Tid:}\newline
09:00\newline 

\hspace{-17pt}\textbf{Sted:}\newline 
Høgskolen i Østfold
\newline

\hspace{-17pt}\textbf{Deltagere:}\newline 
Tom Heine Nätt, Erlend Jacobsen, Markus Holmeset og Jørgen Storm Nielsen\newline

\hspace{-17pt}\textbf{Fra møtet:}\newline

Forside : Tittel : Virker som den fremmhever hva som blir brukt , men burde fremheve hva som kan benyttes. Epost adresser kan flyttes opp til forsiden i fra gruppebeskrivelsen

Innhold : Hvorfor ordner dataservice med bøkene og litt rundt sammarbeid med gyldendal. Skriv det i beskrivelsen av oppgaven om det blir veldig lite,

introduksjon: Beskrivelse av hvordan prosjektet skal løses.

kapittel 2: Omdøpes til “ Bakgrunnsinformasjon”  det bør skrives litt om historikken til fagene.
	      Skrive litt om proggrameringsundervisning gennerelt. 
      2.2 Om det er vesentlig for oppgaven kan det flyttes ned til metodekapittelet.
      2.3 Skriv litt om programmvare som blir benyttet og hvorfor den blir benyttet.

Metode kapittelet : Hvordan skal man komme seg frem til svaret
		 Hvordan skal innhentet informasjon brukes
 Plan for projektets fremdrift
 Upartisk vinkling på oppgaven er viktig
Literatur kan strykes i Metode kapittelet

Diskusjonskapittelet: Ikke del inn i så mange sections, da kan man fort bli låst og strukturert, diskusjonen skal ha en mer åpen flyt. Diskuter tema for tema.

Utrolig mange ting som er feilskrevet , det er viktig å rettskrive oppgaven. ellers kan det blir for mye feil for å klare å rette opp i siste sekund.

Kan flere intervjuer virke mer troverdig for oppgaven. Det er viktig å få en god bredde på oppgaven. Ommfattende innhold er viktig for en troverdig rapport.

Vi burde gå igjennom spørreundersøkelsen for å se om det er noen spesielle svar. og kanskje kontakte de skolene det gjelder.
Det må gjøres mye ut av undersøkelsen, en bra analytisk metode burde benyttes når vi skal bruke den innhentede dataen.
Vi er avhengig av dataen for å finne noe fornuftig , kanskje gruppere de som er negative og positive til forandring. klarer man å se noen sammenhenger mellom hvem som har svar hva, er det likheter.
Hvem svarer på en undersøkelse, er det tilfeldig hvem som svarer eller er det en spessiell gruppe som svarer. Viktig å ta høyde for usikkerhet.
Hvorr mange ble den sendt til


Bakgrunn: Se på artikkler eller “ google scollar”
Skriv en kuleliste under hvert kapittel delkapittel, da er det mye lettere å konvertere hvert pungt om til fornuftig tekst.

Vi trenger mer informasjon om småting, som for eksempel:
	Hva synes elevene om faget.
	Er det tilstrekkelig penger til kursing 
	Hvor mange elever tar kurset. Er det noe trend ?
	Mer intervju.

Hva er vanelig i innføringskurs i proggramering. Er det noe likhet og relevans som kan taes med i oppgaven. Er det bedre for elevene å ha variasjon i flere proggrameringsspråk en kunn ett. \newline



\hspace{-17pt}\textbf{Videre arbeid:}\newline
Lag en plan for arbeidet videre.\newline
Øke effektivitet og enighet om veien videre.\newline
Viktig med god progresjon.


\newpage

