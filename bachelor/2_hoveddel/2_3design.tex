\cleardoublepage
\chapter{Metode}
\label{chap:metode} 

{\color{red} Bevise at vi ikke har gjort ting på måfå / Hvordan kom vi frem til svaret }

I dette prosjektet ble det nødvendig å bruke både kvalitative og kvantitative metoder, siden oppgaven var både å finne ut hva tilstanden var nå og komme med innspill om hva som burde være. Den kvalitative metoden gir grundig informasjon om respondentens tanker og meninger(Dalland, 2012 (It ledelse rapporten)), noe som passer godt for å finne ut om hva lærerne selv synes om dagens situasjon og hvordan de ser for seg fremtiden til faget. Den kvantitave metoden hjelper til å få svar på den andre delen av problemstillingen, nemlig å kartlegge hvilken programvare som faktisk brukes, og få statistikk på dette.

\section{Overordnet metode}
{\color{red} Hvordan vi valgte å organiser oss for å løse oppgaven.} Hva skal være her og hva skal være i prosessrapporten? 
%GPS-GUTTA (systematisk tilnærming)


\section {Informasjonsinnhenting}

\subsection {Programvare}
{\color {red} Programvare: liste over programvare som finnes på dette området }





\subsection {Programmeringsspråk {\color {red} (høgskoler)}}
{\color{red} Skrive om innhenting fra høgskoler og universiteter. }
\subsection {Tidligere eksamensoppgaver}
{\color {red} Hvorfor vi har hentet inn tidligere eksamensoppgaver}

\subsection{Andre land}
{\color {red} Hvordan ting fungerer i andre land?}



\section{Kvalitativ metode}

Den kvalitative metoden baserer seg på små utvalg av respondenter
Prosjektgruppen ønsker å belyse hvordan oppfatningen av faget er hos lærere
Kvaltitativ metode har som hensikt å fange opp mening og opplevelse som ikke lar seg tallfeste eller måle. Den går i dybden og har som formål å få sammenheng og helhet.

\subsection{Halvstrukturert samtaleintervju}
{\color{red} Her må vi forklare at de første halvstrukturerte intervjuene blir brukt som informasjonsinnhenting til hvordan vi skal lage spørreundersøkelsen. Spørsmålet er om det holder at vi Bruker Gløer og Frederik II?  }




\section{Kvantitativ metode}

Kvantitativ metode gir et statisk preg som gjør det vanskelig å fange opp sosiale prosesser (Sander, 2004). 
{\color{red} }


\subsection{Spørreundersøkelse}
{\color{red} Her skriver vi litt om spørreundersøkelse. Her skriver vi inn alle spørsmålene vihar i undersøkelsen, med en beskrivelse på hvorfor vi har valgt å stille nettopp dette spørsmålet, og hvofor vi stiller det slik vi har gjort.}

\subsubsection{Del 1 - Intro}
{\color{red} Introduksjonen til spørreundersøkelsen. Hva vi har skrevet i introen, pluss avkryssing på fylkeskommune og skole det undervisese ved. Hvorfor vi har valgt nettop dette.  }

\subsubsection{Del 2 - IT-1}
{\color{red} Første delene stiller vi spørsmål til IT-1 lærere. Her må vi ha  med at det er todelt (literatur og programvare). Literatur: Hvor mange elever i klassen (for oversikt), Hvilke bøker som benyttes, bruker du eget materiale. 
Programvare: Hvilken programvare og versjon som benyttes, evt annet programvare. Bakgrunn for valg, og hva som tenkes å benyttes neste år. Spørsmålet stilles 4 ganger (bildebehandling, Nettsteder, mutltimedie og databaser. }

\subsubsection{Del 3 - IT-2}
{\color{red} Tilsvarende spørsmål som i del 1, men kun spørsmål om programmering og multimedier.}

\subsubsection{Del 4 - Dagens sitausjon}
{\color{red} Liten del om hvordanm lærerne oppfatter dagens sitausjon, hva de syns om samsvar med læreplan, positivitet til endring i læremidler og vilje til kursing.}





\section{Reliabilitet og validitet}
{ \color {red} (I forbindelse med spørreundersøkelsen vi har utført, er det stor sannsynlighet for at ikke alle har forstått alle spørsmålene osv. Derfor er det viktig å ta med litt om dette i rapporten.)}





