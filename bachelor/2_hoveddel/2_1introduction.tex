\cleardoublepage
\chapter{Introduksjon}
\label{chap:intro}


\section{Prosjektgruppen}


Prosjektgruppen består av tre bachelorstudenter ved Høgskolen i Østfold, prosjekteier og veileder. 
Bachelorstudentene, heretter omtalt som prosjektgruppen, studerer ved linjen for informasjonsteknologi. To av studentene studerer Informasjonssystemer: IT-ledelse, mens en av studentene studerer Ingeniørfag-data. 

\subsection{Jørgen Storm Nielsen}

Jørgen er 3. års student ved Høgskolen i Østfold, avdeling for informasjonssystemer og IT-ledelse.
Har tidligere studert Idrett ved Kristiansand Katedralskole Gimle, og journalistikk ved NLA Mediehøgskolen Gimlekollen. \newline
\textbf{E-post:} jørgen.s.nielsen@hiof.no

\subsection{Erlend Jacobsen}

Erlend er 3.års student ved Høgskolen i Østfold, avdeling for informasjonssystemer og IT-ledelse.
Har tidligere studert Allmenfag på Levanger VGS, og deretter tatt befalskole og jobbet for Forsvaret. \newline
\textbf{E-post:} erlend.h.jacobsen@hiof.no

\subsection{Markus Holmeset}
Markus er 3.års student ved Høgskolen i Østfold, avdeling for (...)
Har tidligere studert Laboratoriefag ved Borg VGS i Sarpsborg og har ffagbrev fra Norske Skog Saugbrugs.\newline
\textbf{E-post:} markus.holmeset@hiof.no\newline
Erlend og Jørgen har tidligere arbeidet sammen i fagene markedsføring og foretaksstrategi,  it og ledelse, software engineering, organisasjonsteori, innføring i bedriftsøkonomisk analyse og prosjektledelse. {\color {red} Eventuelt skrive i flere fag, hvis det blir for mye å ramse opp alle fagene.}
Markus Holmeset studerer til dataingeniør, og har derifra bakgrunn fra grupper i andre fag.



\section{Oppdragsgiver}

Oppdragsgivere i prosjektet er Dataservice og Gyldendal Undervisning. 
Dataservice ble etablert i 1989, og er blant de største i sin bransje i Østfold. Firmaet har base i Halden, men har hele Østfold som sitt markedsområde. 
Selskapet leverer dataprodukter til både private og bedrifter. I tillegg til søsterselskapet
Regnskapsservice, er selskapet også medeier i Odin media, et kommunikasjonsbyrå som leverer tjenester innenfor strategisk rådgivning, grafisk design og webutvikling.\newline
Gyldendal Undervisning er Norges største undervisningsforlag. Forlaget
utgir læremidler til barnehage, grunn- og videregående skole, samt for etter og
videreutdanning innen språkopplæring og andre fagområder.
Gyldendal undervisning utvikler i samarbeid med Dataservice, læremidlene til fagene informasjonsteknologi 1 og 2.

{\color{red} Må skrive litt mer om oppdragsgiver. Forslag til hva mer som kan skrives her?}


\section{Beskrivelse av oppgaven} 
\label{sec:oppgaven}





\subsection{Bakgrunn}
\label{sec:bakgrunn}
Oppdragsgiverne Dataservice og Gyldendal Undervisning, er som tidligere nevnt ansvarlig for utvikling og levering av læremidler i fagene iformasjonsteknologi 1 og 2. 
De nyeste utgavene av lærebøkene som i dag benyttes i de to fagene, er stort sett utgitt i tidsperioden 2011 og 2012 {\color{red} Kilde?}, og nye lærebøker er under utvikling. 
Oppdragsgiver ønsker, i forbindelse med dette å undersøke hvilken programvare og programmeringsspråk som faktisk blir benyttet av lærerne som underviser i fagene.
Prosjektoppgaven vil med bakgrunn i dette, ved hjelp av blant annet en større spørreundersøkelse, kartlegge lærernes bruk av verktøy og programmeringsspråk. 
I tillegg til kartleggigen, ønsker oppdragsgiver en anbefaling fra prosjektgruppen på hvilke programvare som vil være mest fordelaktig å benytte for lærene i undervisningen. 
  
  {\color{red} Førsteutkast av beskrivelse. }




\subsection{Formål}
\label{sec:maal}
Formålet for oppgaven er at prosjektgruppen skal tilegne seg tilstrekkelig informasjon omkring temaet læremidler og programvare i fagene informasjonsteknologi 1 og 2 (heretter omtalt som IT-1 og IT-2), slik at det kan presenteres en grundig kartlegging, og et veloverveid forslag til oppdragsgiver. 

{\color{red} Må fylles ut videre. Skal det benyttes delmål, hovedmål osv?}

\subsection{Problemstilling}

Kartlegging av programvare som i dag benyttes i fagene IT-1 og IT-2, samt en anbefaling angående hvilke verktøy som bør benyttes i undervisningen fremover.


\subsection {Utførelse}

{\color{red}Her skal det stå en beskrivelse av hvordan vi tenker å løse oppgaven.}













