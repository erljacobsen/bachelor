\cleardoublepage
\chapter{Resultater og analyse}

{ \color{red} Skrive om hvordan vi har håndtert den innsamlede datamengden, fra både de halvstrukturerte intervjuene, og spørreundersøkelsen.}

\section{Analyse av data}

{ \color{red} Skrive hvordan vi skal analysere datamengden. Bruke relevant litteratur etc etc..}


\subsection {Intervju med videregåede skoler}

{\color{red} Presentasjon av hvert spørsmål vi har stilt de forskjellige skolene, med litt om hva de har svart på spørsmålene. Typ: Spørsmål 1 = Her svarte den skolen at det var slik, mens den den andre skolen mente det var slik osv. Settes litt opp mot hverandre.
}

\subsubsection{Frederik II VGS}
\subsubsection{Sandvika VGS}
\subsubsection{Elverum VGS}
\subsubsection{Ulstein VGS}
\subsubsection{Sunndal VGS}

\subsection{Spørreundersøkelse}

{\color{red}Presentasjon av hvert spørsmål vi har stilt, med litt om hva de har svart på spørsmålene. Typ: Spørsmål 1 = Her svarte den ene læreren at det var slik, mens den den andre mente det var slik osv. Settes litt opp mot hverandre. }

\subsection{Tilgjengelig programvare }
{\color {red} Her kommer lista over programvare som er tilgjengelig på markedet. }

På markedet i dag finnes det mange typer programvare, som tilbyr de samme funksjonene som programvaren 
lærerene i dag benytter. Prosjektrgruppen ha rlagt mye arbeid i å samle inn flest mulig av disse. 
I tabellene under, vil programvare som er funnet i løpet av denne informasjonsinnhentingen bli presentert.
Det er forskjellige tabeller for hver type programvare (photoshop, DW etc). 
Programmene presenteres i alfabetisk rekkefølge, med en kort beskrivelse om pris og funskjoner.

\textbf {Alternativer til DreamWeaver}

\label{sec:opensource}
\begin{center}
\begin{tabular}{ | m{4cm} | m{2cm}| m{8cm} | } 
 \hline
\textbf{Produkt} & \textbf{Pris} & \textbf{Kommentar} \\ 
\hline
{\color {green} Aloha } & Gratis & \\
\hline
{\color{red} Amaya } & Gratis & \\
\hline
{\color {red}Aptana } & Gratis & \\
\hline
{\color{red} Artisteer4.3} & \$49.95 & \\
\hline 
{\color {green} BlueGriffon} & Gratis & 
\\
\hline
{\color{red} CoffeCup HTML Editor}  & \$69 & 
\\
\hline
{\color{red} Expression Web } & Gratis &  \\
\hline
{\color{red} Google Web Designer} & Gratis &  \\
\hline
{\color{red} Kompozer} & Gratis & \\
\hline
{\color{red} Mirabyte Web Architect } & \$69.95 &  \\
\hline
{\color{red} NamoWebEditor} & Gratis &\\
\hline
{\color{red} NetObjects } & euro130 &  \\
\hline
{\color {green} Open Bexi } & Gratis & 

 \\
\hline
{\color {green} OpenElement} & Gratis & \\
\hline
{\color {green} QuantaPlus} & Gratis &\\
\hline
{\color{red} RapidWeaver (MAC)} & 750 nok &   \\
\hline
{\color{red} StudioLine Web Designer } & \$290 & \\
\hline
{\color{red} Website X5 Evolution 11 } & 643 NOK &  \\
\hline
{\color{red} Website X5 Professional 11} & 1828 NOK & \\
\hline 
{\color{red} Xara web designer 10 } & \$49/\$99 &  \\
\hline

\end{tabular}
\newpage


\textbf{Alternativer til Photo Shop}

\begin{tabular}{ | m{4cm} | m{2cm}| m{8cm} | } 
\hline
\textbf{Produkt} & \textbf{Pris} & \textbf{Kommentar} \\
\hline








\end{tabular}
\end{center}

\newpage
\begin{center}
\begin{tabular}{ | m{4cm} | m{2cm}| m{8cm} | } 
\hline
\textbf{Produkt} & \textbf{Pris} & \textbf{Kommentar} \\
\hline





\end{tabular}
\end{center}

(Som tidligere nevnt foreslår jeg at vi luker ut produktene med bakgrunn i forskjellige kriterier. Da har vi flere produkter som er uaktuelle pga pris, samt noen som er uaktuelle pga mangel på WYSIWYG. Vi sitter da igjen med 5 alternativer som vi kan se nærmere på: Quanta Plus, OpenBexi, BlueGriffon, OpenElement og Aloha.)

\newpage
\subsection{Samling av alternativer til Photo Shop}
\begin{center}
\begin{tabular}{ | m{4cm} | m{2cm}| m{8cm} | } 
\hline
\textbf{Produkt} & \textbf{Pris} & \textbf{Kommentar} \\
\hline
Photo Shop & Kven KVEIT & The world's leader in digital imaging\\
\hline
{\color {green} GIMP } & GRATIS & An open-source alternative to Photoshop that debuted on Unix-based platforms, GIMP stands for GNU Image Manipulation Program. Today it's available in versions for Linux, Windows, and Mac. http://www.creativebloq.com/photoshop/alternatives-1131641\\
\hline
{\color {green} paint.net} & Gratis & Paint.net is a Windows-based alternative to the Paint editor that Microsoft shipped with versions of Windows. Don't let that put you off, though, as it's a surprisingly capable and useful tool, available completely free of charge. The software started out life as a Microsoft-sponsored undergraduate project, and has become an open source project maintained by some of the alumni.\\
\hline
{\color {green} PIXLR }& Gratis & Pixlr claims to be "the most popular online photo editor in the world", which may have something to do with the fact that it's free.\\
\hline
{\color{red} SerifPhotoPlus7 } & £79.99 & Of all the tools featured in this list, PhotoPlus is perhaps the most direct competitor to Photoshop in terms of trying to replicate the different tools in Adobe's software for the PC at a lower price \\
\hline
{\color{red} Aperture } & \$79.99/£54.99 & If you're a photographer, Apple's Aperture is a brilliant alternative to Photoshop. While full of familiar features to Adobe's image editing software, Aperture lacks all the other features - animation, 3D, web, etc - the average photographer never uses, making it much simpler to use.\\
\hline
{\color{red} Acorn (MAC)} & \$29.99/£20.99 & Image editing software Acorn debuted back in 2007 and has provided hobbyists and artists on a budget with a great, affordable alternative to Photoshop ever since. Features of the software include layer styles, non-destructive filters, curves and levels, blending modes and much more.\\
\hline
{\color{red} Sketch } & \$79.99/£54.99 & Sketch by Bohemian Coding is a professional vector graphics app for creatives. With a simple UI, Sketch has many features similar to that of Photoshop and Illustrator, including layers, gradients, colour picker and style presets.\\
\hline
\end{tabular}
\end{center}

\begin{center}
\begin{tabular}{ | m{4cm} | m{2cm}| m{8cm} | } 
\hline
\textbf{Produkt} & \textbf{Pris} & \textbf{Kommentar} \\
\hline

{\color{red} PaintShopPro } & £47.99 & Brought to you by the same software house that produces Painter, Paintshop Pro is a long-standing alternative to Photoshop that offers a huge range of photo-editing and graphics creation tools. The latest version features a streamlined and slick interface with a focus on photography.\\
\hline
{\color{red} Sumopaint } & Free for basic online version, \$4/month for pro version & Sumopaint is a highly capable browser-based image editor. All the standard features you'd expect from a desktop tool are present and correct.\\
\hline
{\color{red} PixelMator} & \$29.99/£20.99 & Pixelmator uses Mac OS X libraries to create fast, powerful image editing tools. As it's built on Mac technologies, it's not available for Windows or Linux, but it does allow the software to integrate seamlessly with the likes of iPhoto and Aperture, as well as iCloud.\\
\hline


\end{tabular}
\end{center}




\section{Kravspesifikasjon}
{ \color{red} Her skriver vi litt om kravspesifikasjon, og hvorfor vi har valgt å bruke dette som grunnlag for å velge programvare i fagene. }


