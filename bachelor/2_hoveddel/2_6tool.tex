\cleardoublepage
\chapter{Benyttet Verktøy}

{\color{green} Her skal vi gi en oversikt og beskrivelse av verktøy vi har benyttet i prosjektet. Og hva vi har benyttet dem til !! 
 }
\newline

\section{Google Docs:}

Google Doc er en rekker Online applikasjoner som kan erstatte de fleste funksjonene i en Office pakke. Dette programmet brukte vi til å felles dele enkle dokumenter, samt sanntid editering av dokumentene.\cite {Googledocs} Det er et veldig nytting applikasjon som flere kan jobbe på sammtidig. Hver bruker får en egen peker som alle kan se til en hver tid. Dokumentet blir automatisk lagret i sanntid så det vil ikke være noen problemer med overlagring og ting som kan forsvinne på grunn av dette.

Vi brukte Google Regneark for å føre timelister i sanntid. Dette er en veldig nyttig funksjon hvor aller kan føre timer i sanntid og det vill alltid være oppdatert for alle. Dette er en veldig enkel og oversiktlig måte å se hva de andre i gruppen har gjort.

Google Document har vi brukt til å føre enkle utkast av tekster samt mail som skulle sendes ut. Da kunne vi enkelt se over og dele informasjon med hverandre. Det ble opprettet et eget mail dokument som vi lagret alt vi hadde sendt ut og til hvem. Ved senere anledninger kunne vi gå inn å hente ut det vi hadde sendt til de forskjellige kontaktpersonene. Svar i fra mailene ble også limt inn i samme dokument.
Møtereferat ble også dokumentert i et eget dokument, hvor det stod en kort avhandling i fra alle møtene.
Hovedbruken til Google Document ble igrunn å kladde tekster og tabeller til det endelige LaTeX Dokumentet. Det var veldig lett å flytte tekst i fra Google Doc til OverLeaf 



Positive sider med google docs kan alle jobbe samtidig og sømløst med felles dokumenter. Man kan også jobbe på dokumenter offline hvis man bruker Google Chrome som nettleser.

Negative sider med google doc's sin data blir lagret på en "remote" server, så du har ikke kontroll på filene og deres tilgjengelighet. Selv om det er liten sansynelighet at serveren går ned og data blir tapt er det en mulighet.


\cite {Googledocs}

\newpage

\section{LaTex}

{\color{green} Inspirasjon hentet ifra http://www.latextemplates.com/why-use-latex og https://www.overleaf.com/benefits
 }

Det ble tatt en avgjørelse på å skrive dokumentet i LaTeX. Et LaTeX dokument ser mye bedre ut en et word dokument, og det er lettere å ha kontroll på store dokumenter. Dette er grunnet LaTex sin "typesetting "
Algorytme som opptimaliserer opsettet av dokumentet.

LaTex er også oprinnelig Gratis hvis man benytter standard programmvare for skriving og kompilering. Det kreves ingen lisens og det fungerer på alle operativsystem. Hvis man lager et .tex dokument på en windows maskin og kompilerer samme filen på et OSX system vil resultatet bli helt likt.
Det er også enkelt å implementere en biblografi manager, noe som regel koster ekstra ved de andre teks editorene.
Med LaTeX trenger man bare å legge ved en .bib fil og implementere denne i main.tex filen. det er også veldig enkelt å maipulere .tex filene, de kan som regel åpnes i en hvilken som helts tekst editor. hele dokumentet er laget i klar tekst som blir gitt en struktur når man kompilerer den. Gratis tekst editor og kompilator finnes på www.tug.org. 

negative sider med LaTex er at det er et nytt system som oppererer med "comand line opperations", så om man ikke er veldig kjent med programerins språket kan det ta noe tid å lære seg dette. Og i akademiske tilfeller er som regel tid en luksus vare.

man må konstant kompilere til nyeste versjon, så om du forandrer en setning må hele dokkumentet kompileres på nytt for å få endringen i dokkumentet. Om man vil forandre strukturen på en side i dokumentet som aviker ifra resten av strukturen kan dette være en veldig tidkrevende prosess.
LaTex er nemmelig optimalisert for å ha en fin standard formatering som man kan forandre på om man vet hva man holder på med.

\subsection{OverLeaf}


Overleaf er et nettbasert tekst editor program med innebygget sanntids kompilering av LaTex dokumenter.



det som er et stort pluss med å bruke denne løsningen er at der er en sterk server som automatisk kompilerer dokumentet til nyeste versjon. Samt at man kan samarbeide med samme prosjekt og skrive på samme dokument.Dette gjør de lettere å jobbe felles i grupper på et prosjekt.
Det er en oversiktilig og fin 







